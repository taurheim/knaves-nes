\documentclass[12pt]{article}

% For Different Coloured Comments
\usepackage{xcolor}

% For better title formatting
\usepackage{titling}

% For version history
\usepackage{vhistory}

% For references
\usepackage{hyperref}

\usepackage{indentfirst}

\usepackage{graphicx}

% To stop tables from being stupid
\usepackage{float}
\restylefloat{table}
\restylefloat{figure}

\setlength{\droptitle}{-10em}

%% Comments
\newif\ifcomments\commentstrue

% Comment Formatting
\ifcomments
\newcommand{\authornote}[3]{\textcolor{#1}{[#3 ---#2]}}
\newcommand{\todo}[1]{\textcolor{red}{[TODO: #1]}}
\else
\newcommand{\authornote}[3]{}
\newcommand{\todo}[1]{}
\fi

% Comment Colours
\newcommand{\wss}[1]{\authornote{magenta}{SS}{#1}}
\newcommand{\hm}[1]{\authornote{blue}{HM}{#1}} %Hediyeh
\newcommand{\tz}[1]{\authornote{blue}{TZ}{#1}} %Tahereh
\newcommand{\pl}[1]{\authornote{blue}{PL}{#1}} %Peng

% Spacing
\setlength{\parindent}{4em}
\setlength{\parskip}{1em}

\begin{document}

% Make the title
\title{MIS for Knaves NES}
\date{\today\\
	{\medskip\small Software Engineering 3XA3, Lab \#3}
}
\author{Niko Savas\\
	\texttt{1300247}
	\and
	Joe Crozier\\
	\texttt{1311502}
	\and
	Sam Nalwa\\
	\texttt{1332792}
}

\maketitle
\clearpage

\tableofcontents
\clearpage

\section{Introduction}
	The following document details the module interface specifications for the implemented modules in KnavesNES. It is intended to ease navigation through the program for design and maintenance purposes. This document is meant to be used in tandem with the System Requirements Specifications and the Module Guide.
\section{Module Heirarchy}
	Because the Knaves NES emulator is simply emulating the CPU of an NES, it doesn't require significant module depth. In the larger scope of an NES emulator, the modules included in the CPU emulator would be lower level modules to the general emulator. For this reason, Knaves NES heirarchy is relatively flat, including only one Level 1 module.

	\begin{table}[H]
		\centering
		\begin{tabular}{p{1.5in} p{1.5in}}
			\hline
			Level 1 & Level 2\\
			\hline
			Command Line Interfacing Module & CPU Module\\
			- & Cartridge Module\\
			- & Memory Module \\
		\end{tabular}
		\caption{Module Heirarchy for Knaves NES}
	\end{table}	

\section{MIS of Command line Interface Module}
	\subsection{Exported Access Programs}
		\begin{table}[H]
			\centering
			\begin{tabular}{p{1.5in} p{1.5in} p{1.5in} p{1.5in}}
				\hline
				\textbf{Name} & \textbf{In} & \textbf{Out} & \textbf{Exceptions}\\
				init & string - filename & - & File Not Found\\
				run & - & - & -\\
				stop & - & - & -\\
			\end{tabular}
			\caption{Exported Access Programs for Command Line Interface Module}
		\end{table}
	\subsection{Interface Semantics}
		\subsubsection{State Variables}
			\textbf{isRunning: } BOOL\\

			\textbf{cycles: } int\\
		\subsubsection{Environment Variables}
			\textbf{\_cpu: } CPU\\

			\textbf{\_memory: } Memory\\
		\subsubsection{Assumption}
			The method calls will be made in the following order: init, run stop.
		\subsubsection{Access Program Semantics}
			Init will initialize the passed cartridge file. Run will start emulation of the CPU, stop will stop the current emulation.


\section{MIS of Memory Module}
	\subsection{Exported Access Programs}
		\begin{table}[H]
			\centering
			\begin{tabular}{p{1.5in} p{1.5in} p{1.5in} p{1.5in}}
				\hline
				\textbf{Name} & \textbf{In} & \textbf{Out} & \textbf{Exceptions}\\
				\hline
				write & unsigned short address, unsigned char value & - & Invalid address, invalid char\\
				read & unsigned short address & unsigned char value & Invalid address\\
				logMemory & - & - & Unable to write file\\
			\end{tabular}
			\caption{Exported Access Programs for Memory Module}
		\end{table}
	\subsection{Interface Semantics}
		\subsubsection{State Variables}
			\textbf{RAM: } char[]
		\subsubsection{Environment Variables}
		\subsubsection{Assumption}
			Reads and writes will be made to addresses within the scope of the memory, otherwise an exception will be called.
		\subsubsection{Access Program Semantics}

\section{MIS of Cartridge Module}
	\subsection{Exported Access Programs}
		\begin{table}[H]
			\centering
			\begin{tabular}{p{1.5in} p{1.5in} p{1.5in} p{1.5in}}
				\hline
				\textbf{Name} & \textbf{In} & \textbf{Out} & \textbf{Exceptions}\\
				\hline
				main & string: FileName & InstructionSet & File Not Found
			\end{tabular}
			\caption{Exported Access Programs for Cartridge Module}
		\end{table}
	\subsection{Interface Semantics}
		\subsubsection{State Variables}
		\subsubsection{Environment Variables}
			\textbf{FileName: } Name of the ROM file to be read
		\subsubsection{Assumption}
		\subsubsection{Access Program Semantics}
			The main function will take the filename given as an input, and read through the ROM file and extract the necessary information and instruction set from it. 

\section{MIS of CPU Module}
	\subsection{Exported Access Programs}
		\begin{table}[H]
			\centering
			\begin{tabular}{p{1.5in} p{1.5in} p{1.5in} p{1.5in}}
				\hline
				\textbf{Name} & \textbf{In} & \textbf{Out} & \textbf{Exceptions}\\
				\hline
				init & - & - & File Not Found\\
				start & - & - & -\\
				reset & - & - & -\\
				executeInstruction & - & unsigned short - cycles & Invalid Instruction\\
				executeInterrupt & enum - interrupt & - & Invalid Interrupt\\
				getSource & Mode- mode & unsigned short - operand & -\\
				checkInterrupts & - & BOOL - interruptRun & -\\
				readAddress & unsigned short - address & unsigned char - value & Invalid Address\\
				pushStack & unsigned char byte & - & -\\
				hasStatus* & unsigned char flag & BOOL - reg\_status & -\\
				updateStatus* & unsigned short val & - & -\\
				funcLoadAccumulator & - & - & -\\
				funcAddWithCarry  & - & - & -\\
				funcBranchNotEqualZero & - & - & -\\
				funcCompareMemory & - & - & -\\
				funcStoreAccumulator & - & - & -\\
				funcTransferAccumulatorToX & - & - & -\\
			\end{tabular}
			\caption{Exported Access Programs for CPU Module}
		\end{table}
	\subsection{Interface Semantics}
		\subsubsection{State Variables}
			\textbf{isRunning: } BOOL\\

			\textbf{reg\_pc: } int\\

			\textbf{reg\_status: } int\\

			\textbf{reg\_x: } int\\

			\textbf{reg\_y: } int\\

			\textbf{instructions: } Enum\\
		\subsubsection{Environment Variables}
			\textbf{\_cpu: } CPU\\

			\textbf{\_memory: } Memory\\
		\subsubsection{Assumption}

		\subsubsection{Access Program Semantics}
			Init will initialize the CPU. Start will begin CPU emulation. Reset will reset the CPU to its original state. ExecuteInstruction will execute the instruction referenced at the current reg\_pc position memory. ExecuteInterrupt will execute the passed interrupt. GetSource will get the source of the operand from the passed mode. CheckInterrupts will check if an interrupt is required to run. ReadAddress will use the Memory module to read the memory value from the passed address. PushStack will push the passed char to the stack. hasStatus* methods will check if the CPU status register includes a given status. ClearStatus* methods will clear the passed flag from the CPU status. UpdateStatus* will update the respective status register. FuncLoadAccumulator will load a value from the referenced address to the accumulator.
\addcontentsline{toc}{section}{Version History}
% Start of the revision history table
\begin{versionhistory}
  \vhEntry{1.0}{Nov 5, 2015}{Niko Savas}{created}
\end{versionhistory}

\end{document}