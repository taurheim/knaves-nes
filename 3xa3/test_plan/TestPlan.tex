\documentclass[12pt]{article}

% For Different Coloured Comments
\usepackage{xcolor}

% For better title formatting
\usepackage{titling}

% For version history
\usepackage{vhistory}

% For references
\usepackage{hyperref}

\usepackage[toc,page]{appendix}
\setlength{\droptitle}{-10em}

%% Comments
\newif\ifcomments\commentstrue

% Comment Formatting
\ifcomments
\newcommand{\authornote}[3]{\textcolor{#1}{[#3 ---#2]}}
\newcommand{\todo}[1]{\textcolor{red}{[TODO: #1]}}
\else
\newcommand{\authornote}[3]{}
\newcommand{\todo}[1]{}
\fi

% Comment Colours
\newcommand{\wss}[1]{\authornote{magenta}{SS}{#1}}
\newcommand{\hm}[1]{\authornote{blue}{HM}{#1}} %Hediyeh
\newcommand{\tz}[1]{\authornote{blue}{TZ}{#1}} %Tahereh
\newcommand{\pl}[1]{\authornote{blue}{PL}{#1}} %Peng

% Spacing
\setlength{\parindent}{4em}
\setlength{\parskip}{1em}

\begin{document}


% Make the title
\title{Test Plan for Knaves NES}
\date{\today\\
	{\medskip\small Software Engineering 3XA3, Lab \#3}
}
\author{Niko Savas\\
	\texttt{1300247}
	\and
	Joe Crozier\\
	\texttt{1311502}
	\and
	Sam Nalwa\\
	\texttt{1332792}
}

\maketitle
\clearpage

\tableofcontents

\listoffigures

\listoftables

\clearpage

\section{Tools for Testing}
\setlength{\parindent}{0cm}
	NESTest\hyperlink{refs}{[1]} - An all inclusive test ROM for NES CPU emulators. The ROM contains test cases for every instruction included in the 6502 instruction set. A link to its source is included in the appendix.

	CC65\hyperlink{refs}{[2]} - A compiler to compile 6502 assembly code into machine code to be read by the emulator. A link to the software is included in the references

	Visual Studio 2015\hyperlink{refs}{[3]} - The IDE used to code and compile Knaves NES. It includes important debugging tools which will be used in the testing process
\section{System Test Cases}
	These tests will be performed in order as they are co-dependant on each other. For example, Test 2, the SDA Instruction Test must utilize the LDA instruction which is tested in Test 1
	\subsection{Test 1 - LDA Instruction Test}
		\textbf{Type: } Dynamic Manual Functional (Black Box) Test\\\\
		\textbf{Summary: } This test is performed to confirm the accuracy and robustness of the LDA instruction. The LDA instruction loads the passed argument into the A register. The instructions are loaded from a custom ROM, the source code of which will be included in appendix \hyperlink{appendix}{[1]}.\\\\
		\textbf{Process}
			\begin{enumerate}
				\item A terminal window is opened in the folder which contains the KnavesNES executable.
				\item The command ‘knavesnes ldatest.nes’ is run.
				\item The ldatest.nes ROM is run by the emulator. Upon its completion, a log file is dumped which contains the CPU’s final memory state.
				\item This state is compared with the desired memory state as outlined in the appendix \hyperlink{appendix}{[2]}.
			\end{enumerate}
		\textbf{Confirmation: } The outputted log file is in agreement with the desired memory state outlined in appendix \hyperlink{appendix}{[2]}.
	\subsection{Test 2 – STA Instruction Test}
		\textbf{Type: } Dynamic Manual Functional (Black Box) Test\\\\
		\textbf{Summary: } This test is performed to confirm the accuracy and robustness of the STA instruction. The STA instruction stores the contents of A into the passed memory address. The instructions are loaded from a custom ROM, the source code of which will be included in appendix \hyperlink{appendix}{[3]}.\\\\
		\textbf{Process}
			\begin{enumerate}
				\item A terminal window is opened in the folder which contains the KnavesNES executable.
				\item The command ‘knavesnes statest.nes’ is run.
				\item The statest.nes ROM is run by the emulator. Upon its completion, a log file is dumped which contains the CPU’s final memory state.
				\item This state is compared with the desired memory state as outlined in the appendix \hyperlink{appendix}{[4]}.
			\end{enumerate}
		\textbf{Confirmation: } The outputted log file is in agreement with the desired memory state outlined in appendix \hyperlink{appendix}{[4]}.
	\subsection{Test 3 – ADC Instruction Test}
		\textbf{Type: } Dynamic Manual Functional (Black Box) Test\\\\
		\textbf{Summary: } This test is performed to confirm the accuracy and robustness of the ADC instruction. The ADC instruction adds the passed value or the contents of the passed memory address to the A register. The instructions are loaded from a custom ROM, the source code of which will be included in appendix \hyperlink{appendix}{[5]}.\\\\
		\textbf{Process}
			\begin{enumerate}
				\item A terminal window is opened in the folder which contains the KnavesNES executable.
				\item The command ‘knavesnes adctest.nes’ is run.
				\item The adctest.nes ROM is run by the emulator. Upon its completion, a log file is dumped which contains the CPU’s final memory state.
				\item This state is compared with the desired memory state as outlined in the appendix \hyperlink{appendix}{[6]}.
			\end{enumerate}
		\textbf{Confirmation: }The outputted log file is in agreement with the desired memory state outlined in appendix \hyperlink{appendix}{[6]}.
	\subsection{Test 4 – TAX Instruction Test}
		\textbf{Type: } Dynamic Manual Functional (Black Box) Test\\\\
		\textbf{Summary: } This test is performed to confirm the accuracy and robustness of the TAX instruction. The TAX instruction transfers the value in the A register to the X register. The instructions are loaded from a custom ROM, the source code of which will be included in appendix \hyperlink{appendix}{[7]}.\\\\
		\textbf{Process}
			\begin{enumerate}
				\item A terminal window is opened in the folder which contains the KnavesNES executable.
				\item The command ‘knavesnes taxtest.nes’ is run.
				\item The taxtest.nes ROM is run by the emulator. Upon its completion, a log file is dumped which contains the CPU’s final memory state.
				\item This state is compared with the desired memory state as outlined in the appendix \hyperlink{appendix}{[8]}.
			\end{enumerate}
		\textbf{Confirmation: } The outputted log file is in agreement with the desired memory state outlined in appendix \hyperlink{appendix}{[8]}.
	\subsection{Test 5 – CMP Instruction Test}
		\textbf{Type: } Dynamic Manual Functional (Black Box) Test\\\\
		\textbf{Summary: } This test is performed to confirm the accuracy and robustness of the CMP instruction. The CMP instruction compares the passed argument with the value in register A. If the values are the same, the Z flag bit is set to 1. If A is greater than or equal to the passed value, the C flag bit is set to 1. The instructions are loaded from a custom ROM, the source code of which will be included in appendix \hyperlink{appendix}{[9]}.\\\\
		\textbf{Process}
			\begin{enumerate}
				\item A terminal window is opened in the folder which contains the KnavesNES executable.
				\item The command ‘knavesnes cmptest.nes’ is run.
				\item The cmptest.nes ROM is run by the emulator. Upon its completion, a log file is dumped which contains the CPU’s final memory state.
				\item This state is compared with the desired memory state as outlined in the appendix \hyperlink{appendix}{[10]}.
			\end{enumerate}
		\textbf{Confirmation: } The outputted log file is in agreement with the desired memory state outlined in appendix \hyperlink{appendix}{[10]}.
	\subsection{Test 6 – BNE Instruction Test}
		\textbf{Type: } Dynamic Manual Functional (Black Box) Test\\\\
		\textbf{Summary: } This test is performed to confirm the accuracy and robustness of the CMP instruction. The BNE instruction checks the value of the Z flag which would have been set by a previous compare operation. If the Z flag is set to 0, the program branches to the passed label. The instructions are loaded from a custom ROM, the source code of which will be included in appendix \hyperlink{appendix}{[11]}.\\\\
		\textbf{Process} 
			\begin{enumerate}
				\item A terminal window is opened in the folder which contains the KnavesNES executable.
				\item The command ‘knavesnes beqtest.nes’ is run.
				\item The beqtest.nes ROM is run by the emulator. Upon its completion, a log file is dumped which contains the CPU’s final memory state.
				\item This state is compared with the desired memory state as outlined in the appendix \hyperlink{appendix}{[12]}.
			\end{enumerate}
		\textbf{Confirmation: } The outputted log file is in agreement with the desired memory state outlined in appendix \hyperlink{appendix}{[12]}.
	\subsection{Test 7 – Loading a Non-Existent ROM file}
		\textbf{Type: } Dynamic Manual Functional (Black Box) Test\\\\
		\textbf{Summary: } This test is used to check the programs ability to detect a file that does not actually exist, but one that is referenced to while running the emulator. When the user loads a file, that does not actually exist on the computer it is being run from the program should state that “No such file exists,” and should prompt the user to try again using another file.\\\\
		\textbf{Process}
			\begin{enumerate}
				\item Open a command window in the directory which contains KnavesNES.
				\item Run the command: “knavesnes xxyyzz.nes” where xxyyzz.nes is a non-existent file
				\item The program should terminate and advise the user to recheck the file name and its existence. 
			\end{enumerate}
	\subsection{Test 8 – Loading a Corrupted ROM file}
		\textbf{Type: } Dynamic Manual Functional (Black Box) Test\\\\
		\textbf{Summary: } This test is used to check if the .nes ROM file the user has requested to open is one that is actually a ROM file, as well as one with a full set of instructions to be further processed. If both those criteria are not met, the emulator should prompt the user to try again with a non-corrupted .nes file.\\\\
		\textbf{Process}
			\begin{enumerate}
				\item Open a command window in the directory which contains KnavesNES.
				\item Run the command: “knavesnes mario.nes” where mario.nes is actually mario.doc with its extension changed.
				\item The program should terminate and advise the user to recheck the file as it is  corrupted/unreadable.
			\end{enumerate}
\section{Non-Functional Tests}
	\subsection{Test 1 - CPU Speed}
		\textbf{Type: } Dynamic Manual Structural (white-box) Test\\\\
		\textbf{Summary: } This test is executed by checking the frequency and cycles per second of the emulated 6502 CPU. This test will give us a more accurate idea on how close our cpu emulation is to the actual 6502 CPU, and is one that must be executed as an emulator relies heavily on the CPU being able to keep up so that there is no delay within the system causing lag later on when the graphics need to be rendered. This test is also required to make sure that the emulated process is not running faster than required, since that would also have a negative impact once the entire emulator is constructed including the graphics rendering module.\\\\
		\textbf{Process: } This test will be done within the code by using the inbuilt system timer. The time taken by each instruction set will be measured, and then the frequency and cycles per second of the processor will be calculated. The test will then be repeated over and over again until the desired emulated processor speed is achieved. 
	\subsection{Test 2 - Usability Test}
		\textbf{Type: } Dynamic Manual Functional (white-box) Test\\\\
		\textbf{Summary: } This test is done to ensure that the software is easy to use. The test requires at least one test subject who is presently unfamiliar with the software, but is capable of basic BASH commands.\\\\
		\textbf{Process: } The test subject is instructed to enter the command knavesnes --help which echo a short help document into the console. The test subject is then instructed to use this information to successfully run a ROM file in the current directory. The test is deemed successful if the subject is able to execute the ROM without any further assistance. 
\section{Automated Testing}
	Automated testing can be done using a bash script, which runs through a list of .nes ROM files, and for each file runs knavesnes, takes a screenshot of the screen, and goes onto the next one. These screenshots can be quickly scrolled through to make sure all the games have executed properly. This list can be as big as a 1000 games, and the test can be automated to run over and over again to make sure it always produces the correct results.
\section{Proof of Concept/Existing Implementation Test}
	For our proof of concept, we will stress test a previously made emulator that we have already compiled using the NESTest rom. By running this previously created emulator with the testing ROM, it will give us a better understanding of the robustness of the emulator and a better idea as to how our system should respond to the testing module.
\section{Schedule of Testing}
	\begin{table}[!ht]
		\centering
		\begin{tabular}{| p{2in} | p{2in} | p{2in} |}
			\hline
			Date of Test & What is Tested & Group Members doing Testing\\
			\hline
			30th October, 2015 & Y.A.N.E. is tested using NESTest to get an idea of robustness & All Members\\
			\hline
			5th November, 2015 & CPU Instructions outlined in the testing document are tested. & Joe\\
			\hline
			10th November, 2015 & knavesnes ROM reading is tested using NESTest to see if all the instructions are readable & Sam\\
			\hline
			15th November, 2015 & knavesnes CPU emulator is tested to ensure that all 6502 instructions work correctly. & Niko\\
			\hline
			25th November, 2015 & knavesnes is tested from start to finish using a CPU test ROM file. & All Members\\
			\hline
		\end{tabular}
		\caption{Schedule of Testing for Knaves NES}
	\end{table}
\clearpage

\addcontentsline{toc}{section}{References}
\hypertarget{refs}{}
\begin{thebibliography}{99}
\bibitem{ref1} \emph{NESTest ROM} - \url{http://nickmass.com/images/nestest.nes}
\bibitem{ref2} \emph{CC65} - \url{http://www.cc65.org/}
\bibitem{ref3} \emph{Visual Studio 2015} - \url{https://www.visualstudio.com/en-us/products/vs-2015-product-editions.aspx}
\end{thebibliography}

\hypertarget{appendix}{}
\begin{appendices}
	\textbf{Appendix 1 }
	\begin{verbatim}
		LDA #$c0
	\end{verbatim}
	\textbf{Appendix 2 }
		\begin{verbatim}
			A = $c0
			X = $00
			Y = $00

			All mem = $00
		\end{verbatim}
	\textbf{Appendix 3}
		\begin{verbatim}
			LDA #$10
			STA $01
			LDA #$20
			STA $02
			LDA #$c0
			STA $c2
			LDA #$fe
			STA $00
			LDA #$19
			STA $00
		\end{verbatim}
	\textbf{Appendix 4}
		\begin{verbatim}
			A = $19 X = $00 Y = $00

			NV-BDIZC
			00110000
		\end{verbatim}
	\textbf{Appendix 5}
		\begin{verbatim}
			LDA #$10
			STA $01
			ADC $01
			LDA #$01
			STA $01
			ADC $12

			LDA #$03
			STA $02
			ADC $23
			LDA #$05
			STA $03
			ADC $34
			LDA #$ff
			STA $03
			ADC $34
		\end{verbatim}
	\textbf{Appendix 6}
		\begin{verbatim}
			A=$c2 X=$00 Y=$00

			NV-BDIZC
			10110000

			0000: 00 01 03 c2 00 00 00 00 00 00 00 00 00 00 00 00 
			0010: 00 00 00 00 00 00 00 00 00 00 00 00 00 00 00 00 
			0020: 00 00 00 00 00 00 00 00 00 00 00 00 00 00 00 00 
			0030: 00 00 00 00 00 00 00 00 00 00 00 00 00 00 00 00 
			0040: 00 00 00 00 00 00 00 00 00 00 00 00 00 00 00 00 
			0050: 00 00 00 00 00 00 00 00 00 00 00 00 00 00 00 00 
			0060: 00 00 00 00 00 00 00 00 00 00 00 00 00 00 00 00 
			0070: 00 00 00 00 00 00 00 00 00 00 00 00 00 00 00 00 
			0080: 00 00 00 00 00 00 00 00 00 00 00 00 00 00 00 00 
			0090: 00 00 00 00 00 00 00 00 00 00 00 00 00 00 00 00 
			00a0: 00 00 00 00 00 00 00 00 00 00 00 00 00 00 00 00 
			00b0: 00 00 00 00 00 00 00 00 00 00 00 00 00 00 00 00 
			00c0: 00 00 00 00 00 00 00 00 00 00 00 00 00 00 00 00 
			00d0: 00 00 00 00 00 00 00 00 00 00 00 00 00 00 00 00 
			00e0: 00 00 00 00 00 00 00 00 00 00 00 00 00 00 00 00 
			00f0: 00 00 00 00 00 00 00 00 00 00 00 00 00 00 00 00
			... 
		\end{verbatim}
	\textbf{Appendix 7}
		\begin{verbatim}
			LDA #$01
			TAX
			LDA #$ff
			TAX
			LDA #$50
			TAX
			LDA #$30
			TAX
			LDA #$21
			TAX
		\end{verbatim}
	\textbf{Appendix 8}
		\begin{verbatim}
			A=$21 X=$21 Y=$00
			NV-BDIZC
			00110000

			All mem = $00
		\end{verbatim}
	\textbf{Appendix 9}
		\begin{verbatim}
			DA #$13
			STA #$00
			CMP #$13
		\end{verbatim}
	\textbf{Appendix 10}
		\begin{verbatim}
			A=$13 X=$00 Y=$00
			NV-BDIZC
			00110011

			0000: 13 00 00 00 00 00 00 00 00 00 00 00 00 00 00 00 
			0010: 00 00 00 00 00 00 00 00 00 00 00 00 00 00 00 00 
			0020: 00 00 00 00 00 00 00 00 00 00 00 00 00 00 00 00 
			0030: 00 00 00 00 00 00 00 00 00 00 00 00 00 00 00 00 
			0040: 00 00 00 00 00 00 00 00 00 00 00 00 00 00 00 00 
			0050: 00 00 00 00 00 00 00 00 00 00 00 00 00 00 00 00 
			0060: 00 00 00 00 00 00 00 00 00 00 00 00 00 00 00 00 
			0070: 00 00 00 00 00 00 00 00 00 00 00 00 00 00 00 00 
			0080: 00 00 00 00 00 00 00 00 00 00 00 00 00 00 00 00 
			0090: 00 00 00 00 00 00 00 00 00 00 00 00 00 00 00 00 
			00a0: 00 00 00 00 00 00 00 00 00 00 00 00 00 00 00 00 
			00b0: 00 00 00 00 00 00 00 00 00 00 00 00 00 00 00 00 
			00c0: 00 00 00 00 00 00 00 00 00 00 00 00 00 00 00 00 
			00d0: 00 00 00 00 00 00 00 00 00 00 00 00 00 00 00 00 
			00e0: 00 00 00 00 00 00 00 00 00 00 00 00 00 00 00 00 
			00f0: 00 00 00 00 00 00 00 00 00 00 00 00 00 00 00 00
		\end{verbatim}
	\textbf{Appendix 11}
		\begin{verbatim}
			  LDA #$02
			  STA $01 
			add:
			  ADC $01 
			  CMP #$06
			  BNE add
			  LDA #$32
			  BRK
		\end{verbatim}
	\textbf{Appendix 12}
		\begin{verbatim}
			A=$32 X=$00 Y=$00
			NV-BDIZC
			00110001

			0000: 00 02 00 00 00 00 00 00 00 00 00 00 00 00 00 00 
			0010: 00 00 00 00 00 00 00 00 00 00 00 00 00 00 00 00 
			0020: 00 00 00 00 00 00 00 00 00 00 00 00 00 00 00 00 
			0030: 00 00 00 00 00 00 00 00 00 00 00 00 00 00 00 00 
			0040: 00 00 00 00 00 00 00 00 00 00 00 00 00 00 00 00 
			0050: 00 00 00 00 00 00 00 00 00 00 00 00 00 00 00 00 
			0060: 00 00 00 00 00 00 00 00 00 00 00 00 00 00 00 00 
			0070: 00 00 00 00 00 00 00 00 00 00 00 00 00 00 00 00 
			0080: 00 00 00 00 00 00 00 00 00 00 00 00 00 00 00 00 
			0090: 00 00 00 00 00 00 00 00 00 00 00 00 00 00 00 00 
			00a0: 00 00 00 00 00 00 00 00 00 00 00 00 00 00 00 00 
			00b0: 00 00 00 00 00 00 00 00 00 00 00 00 00 00 00 00 
			00c0: 00 00 00 00 00 00 00 00 00 00 00 00 00 00 00 00 
			00d0: 00 00 00 00 00 00 00 00 00 00 00 00 00 00 00 00 
			00e0: 00 00 00 00 00 00 00 00 00 00 00 00 00 00 00 00 
			00f0: 00 00 00 00 00 00 00 00 00 00 00 00 00 00 00 00
		\end{verbatim}
\end{appendices}

\clearpage
\addcontentsline{toc}{section}{Version History}
% Start of the revision history table
\begin{versionhistory}
  \vhEntry{1.0}{Oct 23, 2015}{Niko Savas}{created}
\end{versionhistory}
\end{document}