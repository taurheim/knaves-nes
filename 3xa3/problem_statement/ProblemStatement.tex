\documentclass[12pt]{article}

\usepackage{xcolor} % for different colour comments
\usepackage{titling}

\setlength{\droptitle}{-10em}   % This is your set screw


%% Comments
\newif\ifcomments\commentstrue

\ifcomments
\newcommand{\authornote}[3]{\textcolor{#1}{[#3 ---#2]}}
\newcommand{\todo}[1]{\textcolor{red}{[TODO: #1]}}
\else
\newcommand{\authornote}[3]{}
\newcommand{\todo}[1]{}
\fi

%% Comment Colours
\newcommand{\wss}[1]{\authornote{magenta}{SS}{#1}}
\newcommand{\hm}[1]{\authornote{blue}{HM}{#1}} %Hediyeh
\newcommand{\tz}[1]{\authornote{blue}{TZ}{#1}} %Tahereh
\newcommand{\pl}[1]{\authornote{blue}{PL}{#1}} %Peng

\setlength{\parindent}{4em}
\setlength{\parskip}{1em}

\begin{document}

\title{Problem Statement for Knaves NES}
\date{\today\\
	{\medskip\small Software Engineering 3XA3, Lab \#3, Group \#10}
}
\author{Niko Savas\\
	\texttt{1300247}
	\and
	Joe Crozier\\
	\texttt{1311502}
	\and
	Sam Nalwa\\
	\texttt{1332792}
}
	
\maketitle

KnavesNES aims to emulate the Mos Technology 6502 microprocessor, specifically, the exact model used in the Nintendo Entertainment System (1983). Emulating a CPU is a challenging problem which requires significant knowledge of computer hardware, assembly programming, and memory management.

The context for this project's creation is the continued interest in emulating and learning about old hardware as a pursuit of programmers and hobbyists. As computers get more and more complex, understanding them in detail becomes more and more difficult. Learning about how simpler architecture functions allows for a better understanding by using concrete examples. With this in mind, Knaves NES is built keeping in mind the two main stakeholders.

The first important stakeholder is the amateur software developer, who would require an environment to test his 6502 assmelby code. Such code cannot be easily compiled in a normal operating system, so an emulated system must be used. By learning to develop software in an assembly language, the amateur developer could gain a greater understanding of the way computers compile and process higher level languages.

The other stakeholder that Knaves NES is built for is the computer engineer who wants to understand at a deeper level how instructions are run by the 6502 architecture. The code behind Knaves NES is structured and written to make it simple to understand for an amateur with basic knowledge of CPUs. By releasing the source code for Knaves NES on GitHub, it would be simple for any parties interested to inspect the code base and either contribute to it or learn from it.

With these major stakeholders in mind, it is important that the Knaves NES CPU should be able to process all of the documented opcodes for the 6502 architecture as the physical CPU in an NES would. The CPU will read the opcodes from an NES ROM file and process them in the same way that the 6502 processor would. The CPU will manage 2kB of RAM that can be modified through instructions executed from the ROM. Emulating a CPU serves as an interesting and challenging problem which will educate its developers in a variety of topics.

\end{document}