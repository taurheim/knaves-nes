\documentclass[12pt]{article}

\usepackage{xcolor} % for different colour comments
\usepackage{titling}

\setlength{\droptitle}{-10em}   % This is your set screw


%% Comments
\newif\ifcomments\commentstrue

\ifcomments
\newcommand{\authornote}[3]{\textcolor{#1}{[#3 ---#2]}}
\newcommand{\todo}[1]{\textcolor{red}{[TODO: #1]}}
\else
\newcommand{\authornote}[3]{}
\newcommand{\todo}[1]{}
\fi

%% Comment Colours
\newcommand{\wss}[1]{\authornote{magenta}{SS}{#1}}
\newcommand{\hm}[1]{\authornote{blue}{HM}{#1}} %Hediyeh
\newcommand{\tz}[1]{\authornote{blue}{TZ}{#1}} %Tahereh
\newcommand{\pl}[1]{\authornote{blue}{PL}{#1}} %Peng

\setlength{\parindent}{4em}
\setlength{\parskip}{1em}

\begin{document}

\title{Problem Statement for Knaves NES}
\date{\today\\
	{\medskip\small Software Engineering 3XA3, Lab \#3}
}
\author{Niko Savas\\
	\texttt{1300247}
	\and
	Joe Crozier\\
	\texttt{1311502}
	\and
	Sam Nalwa\\
	\texttt{1332792}
}
	
\maketitle

KnavesNES aims to emulate the Mos Technology 6502 microprocessor, specifically, the exact model used in the Nintendo Entertainment System (1983). Emulating a CPU is a challenging problem which requires significant knowledge of computer hardware, assembly programming, and memory management.

 The emulated CPU should be able to process the same operation codes, represented as hexadecimal numbers, as the physical CPU. The CPU will manage its own RAM, in this case 2kB. The CPU must be able to process instructions from a ROM file, as well as store the contents of this ROM file in memory. Tests are available as ROM files with sample instructions meant to test the fidelity of the CPU. The context of this project is that of the consistent pursuit by programmers to emulate pieces of hardware using software.

The main stakeholder in this project is an amateur software developer who requires an environment to test his 6502 assembly code with. Such code cannot be easily compiled in a normal operating system, so an emulated system must be used. By learning to develop software in an assembly language, the amateur developer could gain a greater understanding of the way computers compile and process higher level languages.

 An important environment for this software would be that of education. In a computer science class teaching students about computer hardware and processor infrastructure, students are often taught the basics of assembly programming. By providing them a means to actually run and debug their assembly code (in this case, 6502), students may find the language easier to grasp. This means that the students and the instructors can act as important stakeholders in this project. Emulating a CPU serves as an interesting and challenging problem which will educate its developers in a variety of topics.

\end{document}