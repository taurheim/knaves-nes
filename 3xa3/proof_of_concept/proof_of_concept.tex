\documentclass[12pt]{article}

\usepackage{xcolor} % for different colour comments
\usepackage{titling}
\usepackage{hyperref} %for url displaying

\setlength{\droptitle}{-5em}   % This is your set screw


%% Comments
\newif\ifcomments\commentstrue

\ifcomments
\newcommand{\authornote}[3]{\textcolor{#1}{[#3 ---#2]}}
\newcommand{\todo}[1]{\textcolor{red}{[TODO: #1]}}
\else
\newcommand{\authornote}[3]{}
\newcommand{\todo}[1]{}
\fi

%% Comment Colours
\newcommand{\wss}[1]{\authornote{magenta}{SS}{#1}}
\newcommand{\hm}[1]{\authornote{blue}{HM}{#1}} %Hediyeh
\newcommand{\tz}[1]{\authornote{blue}{TZ}{#1}} %Tahereh
\newcommand{\pl}[1]{\authornote{blue}{PL}{#1}} %Peng

\setlength{\parindent}{4em}
\setlength{\parskip}{1em}

\begin{document}

\title{Proof of Concept for Knaves NES}
\date{\today\\
	{\medskip\small Software Engineering 3XA3, Lab \#3, Group \#10}
}
\author{Niko Savas\\
	\texttt{1300247}
	\and
	Joe Crozier\\
	\texttt{1311502}
	\and
	Sam Nalwa\\
	\texttt{1332792}
}
	
\maketitle

KnavesNES will be created using C++ to emulate a MOS Technology 6502 CPU, as found in an NES console. The program will read instructions from NES ROM files and execute them on the virtual CPU. The part of the implementation that will be challenging will be emulating the CPU as it will require a fair bit of learning related to assembly programming and memory management. 

 Testing should not be difficult, since we will be using ROMS which have specifically been designed to test emulated CPUs. Executing the program with these ROMS as inputs will give us an accurate result if our emulator is functioning properly.

We will be using the Standard C++ library for this project, which is easily installble on any computer as it is widely available on the internet. 

 Portability should not be a concern, since the project will be coded in C++, which can be complied and run on Windows, Macs and Linux computers. 

We have already successfully compiled an existing NES emulator called Y.A.N.E. \href{http://alike.se/yane/}{(Link)}, which is essentially a MOS Technology 6502 CPU emulator with added graphics and sound emulation. We believe that we can use this working model to give us a better understanding of how our emulator will work.

\end{document}